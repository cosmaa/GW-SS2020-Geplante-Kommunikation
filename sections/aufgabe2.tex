\newpage
\section{Szenarien/Risiken und Strategie/Taktik}

\subsection{Ablauf des Gesprächs}
\begin{itemize}
   \item Gespräch wurde vorher angekündigt über Whats-App:
    \begin{MyColorPar}{black}
        \begin{center}
            \begin{minipage}{25em}
               HUHU, ihr Nasen ich hab eine Idee für das Spülmaschine-Ausräumen, ich erzähl's euch heute Abend.
            \end{minipage}
        \end{center}
    \end{MyColorPar}
    \item alle sitzen zusammen in der Küche:
\end{itemize}

\hspace{\dimexpr-\fboxrule-\fboxsep\relax}\fbox{
    \begin{minipage}[t][11cm][t]{0.49\textwidth}
        \centering\textbf{SZENARIO 1}
            \begin{itemize}
                \item alle hatte einen normalen Arbeitstag 
                \item alle haben gegessen
                \item sind in entspannter Erwartung
                \item ich er kläre das Werkzeug
                \item ich versichere mich über ein allgemeines Verständnis über das Werkzeug mit Hilfe eines live Beispiels
                \item Fragen/Kritik/Argumentation: 
                \begin{itemize}
                    \item sorgfältiges Abwägen von Pro und Kontra
                    \item Ringen um Positionen oder Überzeugungen
                \end{itemize}
                \item Einigung
            \end{itemize}
    \end{minipage}
}\hspace{2mm}
\hspace{\dimexpr-\fboxrule-\fboxsep\relax}\fbox{
    \begin{minipage}[t][11cm][t]{0.49\textwidth}
        \centering\textbf{SZENARIO 2}
            \begin{itemize}
                \item alle hatten einen anstrengenden Arbeitstag
                \item es ist später als gedacht
                \item alle sind müde
                \item sie wollen aber trotzdem noch hören was ich mir ausgedacht habe
                \item ich erkläre das Werkzeug
                \item es gibt keine wirkliche Resonanz 
                \item lenken sich immer wieder mit anderen Gesprächen ab 
                \item es gibt zerstreute Gegenargumente und Positionen
            \end{itemize}
            \begin{mybox}
                \begin{center}
                    Hier vielleicht ausweichen auf Alternativ-Szenario 
                \end{center}
            \end{mybox}
    \end{minipage} 
}

\clearpage
\subsection{Risiken}

\begin{table*}[h]
\centering
\begin{tabular}{p{0.45\linewidth}p{0.45\linewidth}}
\toprule[1.5pt]
\textbf{Risiko} & \textbf{Reaktion} \\
\midrule
Alle beharren auf ihre Standpunkte  & herausarbeiten wo wir uns einig sind und Gemeinsamkeiten schaffen \\
\midrule
Verständnis für die Notwendigkeit von Regeln in einer WG   & Argumente bereit legen \\
\midrule
Aneinander vorbei reden  & Zuhören und Nachfragen um gemeinsames Verständnis zu schaffen \\
\midrule
Keine Einigung auf eine {\glqq Regelung\grqq} / angestrengtes diskutieren  & Ausweichen auf alternativ Szenario \\
\bottomrule[1.25pt]
\end{tabular}
\caption{Risiko-Einschätzung}
\end{table*}


\subsection{Alternativ-Szenario}
Kriterien für den Übergang auf das Alternativ-Szenario:
\begin{itemize}
    \item hitzige Diskussion
    \item Passivität Einzelner Teilnehmer (aufs Handy gucken, immer wieder aus der Küche gehen, ablenken mit andere Gesprächsthemen)
    \item ich fühle mich unwohl und persönlich angegriffen 
    \item jemand agiert irrational/ unsachlich
\end{itemize}
\begin{center}
    \hspace{\dimexpr-\fboxrule-\fboxsep\relax}\fbox{
    \begin{minipage}[b][5cm][t]{0.75\textwidth}
         \centering \textbf{ALTERNATIV SZENARIO}
          \begin{itemize}
            \item ich gehe auf die Situation ein und erkläre dass sie ja gerne mal drüber nachdenken können bis zum nächsten Treffen, oder vielleicht haben sie ja selbst Vorschläge zum Thema
            \item wir vereinbaren einen neuen Termin (evtl. am Wochenende)
            \item vielleicht ein genaues Zeitfenster bestimmen (Bsp. max 1/2 Stunde oder Stunde)
        \end{itemize}
    \end{minipage} 
}
\end{center}
\clearpage

\subsection{Strategie/Taktik}
Weitestgehend würde ich die Verhandlung nach dem Modell \footnote{\ SoSe2020 U.Doleschel - Geplante Kommunikation Folie 50} Knochs und Schwedhelms durchführen. Damit wir alle mit einem guten Gefühl aus dem Gespräch gehen, ist es vielleicht auch hilfreich das Modell auszudrucken und immer wieder ab zu gleichen ob sich alle gut fühlen mit den Beschlüssen und keiner sich übergangen fühlt.
\begin{figure}[h!]
\centering
\includegraphics[width=400pt]{pics/VerhandlungsModell.png}
\end{figure}
\captionof{figure}{ Knoch und Schwedhelm Verhandlungs-Modell }

    