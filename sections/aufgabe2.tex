\section{Szenarien und Risiken}

\subsection{Ablauf des Gesprächs}
\begin{MyColorPar}{ForestGreen}
Skizzieren Sie den ihrer Ansicht nach wahrscheinlichsten Ablauf des Gespräches (z.B. Feilschen um jeden Cent, sorgfältiges Abwägen von Pro und Kontra, zähes Ringen um Positionen oder Überzeugungen, etc.).
\end{MyColorPar}

\begin{itemize}
    \item 
\end{itemize}

\subsection{Ablauf des Gesprächs}
\begin{MyColorPar}{ForestGreen}
Welche Risiken (z.B. unsachliche Unterstellungen, Persönliche Angriffe, Fehleinschätzungen, in einen Konflikt hineingezogen werden, mangelnder Einigungs-Wille, etc.) sehen Sie in den Gesprächs-Phasen und wie können Sie damit umgehen?
\end{MyColorPar}

\begin{itemize}
    \item 
\end{itemize}

\subsection{Alternativ-Szenario}
\begin{MyColorPar}{ForestGreen}
Skizzieren Sie mindestens ein Alternativ-Szenario und definieren Sie die Kriterien für den Übergang auf das Alternativ-Szenario.
\end{MyColorPar}

Kriterien für den Übergang auf das Alternativ-Szenario:


\begin{itemize}
    \item 
\end{itemize}
\begin{MyColorPar}{ForestGreen}

Welche Risiken (z.B. unsachliche Unterstellungen, Persönliche Angriffe, Fehleinschätzungen, in einen Konflikt hineingezogen werden, mangelnder Einigungs-Wille, etc.) sehen Sie in den Gesprächs-Phasen und wie können Sie damit umgehen?
Skizzieren Sie mindestens ein Alternativ-Szenario und definieren Sie die Kriterien für den Übergang auf das Alternativ-Szenario.
\end{MyColorPar}

