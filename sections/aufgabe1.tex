\section{Die Ausgangslage} 

\subsection{Gegenstand des Kommunikations-Plans}

In einer 3 Personen Wohn-Gemeinschaft soll die Verantwortlichkeit für die Aufgabe "Geschirrspüler ausräumen" durch ein neues Werkzeug reguliert werden.\newline

Werkzeug:\newline
Ein einfacher Pappe-Kreis, der wie eine Torte in 3 gleich große Teile unterteilt ist. In jedem  dieser Drittel steht ein Name eines Mitbewohners. Ein Zahnstocher-Fähnchen markiert den aktuellen Verantwortlichen für die Aufgabe, sobald diese erfüllt ist wird das Fähnchen vom aktuellen, im Uhrzeigersinn, an den nächsten Verantwortlichen weiter gesteckt.\newline

Die Meinungen, zu diesem Tool, gingen weit auseinander. Obwohl es für mich einleuchtend und hilfreich wirkte, kam es bei meinen Mitbewohnern eher gar nicht oder störend an. Daher finde ich es spannend mich mit dieser Fehlgeschlagenen Kommunikation noch einmal, geplant auseinander zu setzen.

\subsection{Die Spieler}

\textbf{Gesprächspartner und Situation}

\begin{itemize}
    \item Mitbewohner Tilo, Robert und ich 
    \item wohnen seit 2 Jahren zusammen
    \item die Grundstimmung ist locker, lustig, offen
    \item Es gibt keine konkrete Regelung in der WG. Es wird allgemein angenommen, dass sich alle im gleichen Maße verantwortlich für diese Aufgabe fühlen
    \item Wer macht was, so objektiv betrachtet wie möglich, in absteigender  Beteiligung:
    \begin{itemize}
         \item Tilo räumt immer dann den Geschirrspüler aus wenn er in der Küche ist und es sieht, also objektiv betrachtet erledigt er die Aufgabe am häufigsten
        \item ich erledige die Aufgabe grundsätzlich dann, wenn ich Zeit habe, dabei spielt es auch eine Rolle ob ich es gerade die Tage auch schon einmal gemacht habe 
        \item Robert Sieht die Aufgabe nicht und erledigt sie dann wenn er den Geschirrspüler braucht
    \end{itemize}
    \item die Dringlichkeit der Regelung kam auf als Tilo seine Unzufriedenheit äußerte Zitat: "Wie kann es sein, dass seit 3 Tagen der Geschirrspüler nicht ausgeräumt wird"
    \item Tilo und Robert wollen keine Regeln sie wollen, dass es so funktioniert  jeder sollte Verantwortung übernimmt
\end{itemize}

\textbf{emotionales Engagement}

\begin{itemize}
    \item Regeln sind für mich ein Tool, um die Grenzen in der WG abzustecken
    \item ich bin die einzige die Regel für sinnvoll hält
    \item Ich versuche das Werkzeug zu Verkaufen
    \item Ich bin emotional engagiert
\end{itemize}

\subsection{Ziel }


Ich möchte mein eigenes Wohlbefinden in der WG steigern. Denn durch den ungleichen Aufwand kommt es bei mir zu schlechtem Gewissen oder Trotz oder Unmut. Ich denke durch das Werkzeug kann sich jeder zurück lehnen und muss nur dann Aktiv werden, wenn das Fähnchen bei ihm/ihr angelangt ist. Die Aufgabe wäre dann klar adressiert.\newline

\textbf{Rote-Linie} \newline
Ich möchte eine geeignetere Regelung finden als die Bestehende.
Daher möchte ich aus diesem Gespräch mit mindesten einem neuen Ansatz als "jeder macht es". Schön wäre es, wenn wir das Werkzeug zeitweise testen.\newline

\textbf{Best-Case} \newline
Volle Akzeptanz des Werkzeugs und positive Kritik.
