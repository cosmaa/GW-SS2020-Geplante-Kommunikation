\section{Informations-Management}

\subsection{Wo suchen ich nach Argumenten?}
- In der Kommunikation mit der WG \\
- Im Internet nach der Korrektheit meiner Argumente\\
- verweisen auf Beispiel\\

\clearpage
\subsection{Mit welchen Gegenargumenten sollte ich rechnen?}
\begin{table*}[h!]
\centering
\begin{tabular}{p{0.01\linewidth}p{0.24\linewidth}p{0.7\linewidth}}
{\Large \textbf{  }} & {\Large \textbf{Gegen-Argument}} & {\Large \textbf{Argument}} \\
\addlinespace
\toprule[1.5pt]
\textbf{1} & \textbf{glaubt nicht, dass das Werkzeug bei uns funktioniert}   & Herausfinden was \textbf{ nicht funktionieren wird} / Fragen stellen \\
\midrule
 & würde nicht drauf achten  & was hindert dich daran darauf zu achten kann man da entgegenwirken - oder es ist einfach eine Trotz Reaktion gegen Regeln allgemein, siehe \textbf{Gegen-Argument 3}\\
\midrule
 & würde vergessen das Fähnchen weiter zu stecken  & was hindert dich daran darauf zu achten kann man da entgegenwirken - oder es ist einfach eine Trotz Reaktion gegen Regeln allgemein, siehe \textbf{Gegen-Argument 3}\\
\midrule
 & was passiert, wenn jemand im Urlaub ist  & dann wird derjenige übersprungen\\
\bottomrule[1.2pt]
\textbf{2} &  \textbf{appelliert immer wieder an die Eigenverantwortung}  & Beweis: die aktuellen Situation, dass es nicht funktioniert, weil jeder ein anderes Bild von der Situation hat, bzw jeder ein anderes Maß Verantwortung empfindet für die Aufgabe. Um aber eine gerechtere Situation zu schaffen muss jedem klar zugewiesen werden wann er/sie an der Reihe ist. Hierzu braucht es Regeln, , siehe \textbf{Gegen-Argument 3}\\
\bottomrule[1.2pt]
\textbf{3} & \textbf{findet so viele Regeln nervig und anstrengend}  & ausführen warum regeln wichtig sind in einer Gemeinschaft:

{\color{blue} Die staatlichen Gesetze sind dafür da, dass sie ein friedliches Zusammenleben in einer großen Gemeinschaft regeln und erleichtern. Sie legen fest, was man tun muss, tun darf oder nicht tun darf. }

{\color{violet}Regeln zeigen dem Mitbewohner, was im Zusammenleben mit anderen von ihm erwartet wird und er/sie selbst von den anderen erwarten darf}\\
\bottomrule[1.6pt]
\bottomrule[1.6pt]
\end{tabular}
\caption{Argumentation bei Gegen-Argumenten}
\end{table*}